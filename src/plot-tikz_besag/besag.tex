\documentclass[letterpaper,twocolumn,10pt]{article}

\usepackage[dvipsnames]{xcolor}
\usepackage{tikz}
\usetikzlibrary{backgrounds}
\usetikzlibrary{arrows,shapes}
\usetikzlibrary{tikzmark}
\usetikzlibrary{calc}

\usepackage{amsmath}
\usepackage{amsthm}
\usepackage{amssymb}
\usepackage{mathtools, nccmath}
\usepackage{wrapfig}
\usepackage{comment}

% To generate dummy text
\usepackage{blindtext}

%\usepackage[pdftex]{graphicx}
\usepackage{graphicx}

% for custom commands
\usepackage{xspace}

% table alignment
\usepackage{array}
\usepackage{ragged2e}
\newcolumntype{P}[1]{>{\RaggedRight\hspace{0pt}}p{#1}}
\newcolumntype{X}[1]{>{\RaggedRight\hspace*{0pt}}p{#1}}

% color box
\usepackage{tcolorbox}

% for tikz
\usepackage{tikz}
%\usetikzlibrary{trees}
\usetikzlibrary{arrows,shapes,positioning,shadows,trees,mindmap}
% \usepackage{forest}
\usepackage[edges]{forest}
\usetikzlibrary{arrows.meta}
\colorlet{linecol}{black!75}
\usepackage{xkcdcolors} % xkcd colors

% for colorful equation
\usepackage{tikz}
\usetikzlibrary{backgrounds}
\usetikzlibrary{arrows,shapes}
%\usetikzlibrary{tikzmark}
\usetikzlibrary{calc}
% Commands for Highlighting text -- non tikz method
\newcommand{\highlight}[2]{\colorbox{#1!17}{$\displaystyle #2$}}
%\newcommand{\highlight}[2]{\colorbox{#1!17}{$#2$}}
\newcommand{\highlightdark}[2]{\colorbox{#1!47}{$\displaystyle #2$}}

% my custom colors for shading
\colorlet{mhpurple}{Plum!80}

% Commands for Highlighting text -- non tikz method
\renewcommand{\highlight}[2]{\colorbox{#1!17}{#2}}
\renewcommand{\highlightdark}[2]{\colorbox{#1!47}{#2}}

% Some math definitions
\newcommand{\lap}{\mathrm{Lap}}
\newcommand{\pr}{\mathrm{Pr}}

\newcommand{\Tset}{\mathcal{T}}
\newcommand{\Dset}{\mathcal{D}}
\newcommand{\Rbound}{\widetilde{\mathcal{R}}}

\begin{document}

\begin{wrapfigure}{l}{0.5\columnwidth}
    \vspace{\baselineskip}
    \begin{equation}
    \label{eq:laplace_density}
        \lap (x\ |\ \tikzmarknode{u}{\highlight{red}{$\mu$}}, \tikzmarknode{b}{\highlight{blue}{b}}) = \frac{1}{2b} \mathrm{exp}(-\frac{|x-\mu|}{b})
    \end{equation}
    \begin{tikzpicture}[overlay,remember picture,>=stealth,nodes={align=left,inner ysep=1pt},<-]
        % For "mu"
        \path (u.north) ++ (0,2em) node[anchor=south west,color=red!67] (scalep){\textbf{location parameter, mean}};
        \draw [color=red!57](u.north) |- ([xshift=-0.3ex,color=red]scalep.south east);
        % For "b"
        \path (b.south) ++ (0,-1.5em) node[anchor=north west,color=blue!67] (mean){\textbf{$b >0$, scale parameter}};
        \draw [color=blue!57](b.south) |- ([xshift=-0.3ex,color=blue]mean.south east);
    \end{tikzpicture}
    \vspace{0.5\baselineskip}
    \caption{An example in the single column format using the wrapfig construct.}
    \vspace{0.5\baselineskip}
\end{wrapfigure}

\end{document}
